%*******************************************************
% Abstract
%*******************************************************
%\renewcommand{\abstractname}{Abstract}
%\pdfbookmark[1]{Abstract}{Abstract}
% \addcontentsline{toc}{chapter}{\tocEntry{Abstract}}
\begingroup
\let\clearpage\relax
\let\cleardoublepage\relax
\let\cleardoublepage\relax
\chapter*{Abstract}

Gravitational waves have recently emerged as a novel messenger in astronomy, providing profound insights into the evolution of the cosmos: Last year, several \acp{PTA} announced the detection of a \ac{GWB} at nHz frequencies, marking a significant milestone in this field. This discovery can be interpreted as a signal emitted by the inspiral of supermassive black holes during hierarchical structure formation. More excitingly, the signal’s origin could alternatively predate recombination and Big Bang nucleosynthesis. Such an explanation would require new physics in order to explain the strong dynamics in the primordial plasma above the MeV temperature scale necessary to emit those gravitational waves.

In this thesis, we explore the possibility that \acp{PTA} could have detected a cosmological phase transition within a dark sector. A better fit to the data than that for supermassive black hole binaries can be found if the dark sector hosting the transition decays before the onset of nucleosynthesis.

Furthermore, it is plausible that \acp{PTA} have instead observed a \ac{GWB} emitted by the inspiral of supermassive \textit{primordial} black holes. Our research indicates that these primordial black holes can account for the observed signal only if they are initially clustered rather than homogeneously distributed.

Looking towards future gravitational wave observatories, we examine the formidable opportunity offered by \acs{LISA}, that is sensitive to mHz frequencies, to probe \acsp{GWB} emitted at temperatures of the primordial plasma around the $100 \, \text{GeV}$ scale, aligning with the freeze-out epoch of a \acs{WIMP} dark matter candidate. Our findings suggest that if \acs{LISA} detects such a cosmological \ac{GWB}, it would hint towards a dark sector phase transition producing the relic dark matter abundance, consistent with cosmological and astrophysical observations.

Our results underscore the significance of gravitational wave observations in unveiling the dynamics of the early universe and provide new pathways for understanding the fundamental nature of dark matter and the cosmological history of the universe.


\newpage
\begin{otherlanguage}{ngerman}
%\pdfbookmark[1]{Zusammenfassung}{Zusammenfassung}
\chapter*{Zusammenfassung}
Gravitationswellen haben sich jüngst als neuartige Boten in der Astronomie herausgestellt und liefern tiefgreifende Einblicke in die Entwicklung des Kosmos: Letztes Jahr kündigten mehrere \acp{PTA} die Entdeckung eines Gravitationswellenhintergrunds bei nHz-Frequenzen an, was einen bedeutenden Meilenstein in diesem Bereich darstellt. Diese Entdeckung kann als Signal interpretiert werden, das durch die Verschmelzung von supermassiven Schwarzen Löchern während der hierarchischen Strukturentstehung emittiert wurde. Noch spannender ist die Möglichkeit, dass der Ursprung des Signals alternativ vor der Rekombination und der Urknall-Nukleosynthese liegt. Eine solche Erklärung würde neue Physik erfordern, um die starke Dynamik im primordialen Plasma über der MeV-Temperaturskala zu erklären, die zur Emission von diesen Gravitationswellen notwendig ist.

In dieser Dissertation untersuchen wir die Möglichkeit, dass \acp{PTA} einen kosmologischen Phasenübergang innerhalb eines dunklen Sektors detektiert haben könnten. Eine bessere Übereinstimmung mit den Daten als bei supermassiven Schwarzen-Loch-Binärsystemen kann gefunden werden, wenn der dunkle Sektor, der den Übergang beherbergt, vor dem Beginn der Nukleosynthese zerfällt.

Darüber hinaus ist es plausibel, dass \acp{PTA} stattdessen einen Gravitationswellenhintergrund beobachtet haben, der durch die Verschmelzung von supermassiven \textit{primordialen} Schwarzen Löchern emittiert wurde. Unsere Forschung zeigt, dass diese primordialen Schwarzen Löcher das beobachtete Signal nur dann erklären können, wenn sie anfangs geklumpt und nicht homogen verteilt sind.

Mit Blick auf zukünftige Gravitationswellen-Observatorien untersuchen wir die interessante Koinzidenz, dass \acs{LISA} für mHz-Frequenzen empfindlich sein wird, die Gravitationswellen-Hintergründen entsprechen, die bei Temperaturen des primordialen Plasmas um die $100 \, \text{GeV}$-Skala emittiert werden und mit der Freeze-out-Epoche eines \acs{WIMP}-Dunkle-Materie-Kandidaten übereinstimmen. Unsere Ergebnisse legen nahe, dass, wenn \acs{LISA} einen solchen kosmologischen Gravitationswellenhintergrund detektiert, dies auf einen Phasenübergang in einem dunklen Sektor hinweisen würde, der die Dunkle Materie-Menge  produziert, die mit kosmologischen und astrophysikalischen Beobachtungen übereinstimmt.

Unsere Ergebnisse unterstreichen die Bedeutung von Gravitationswellenbeobachtungen für die Enthüllung der Dynamik des frühen Universums und bieten neue Wege zum Verständnis der grundlegenden Natur der Dunklen Materie und der kosmologischen Geschichte des Universums.



\end{otherlanguage}

\endgroup

\vfill
