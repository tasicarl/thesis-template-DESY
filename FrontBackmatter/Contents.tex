%*******************************************************
% Table of Contents
%*******************************************************
\pagestyle{scrheadings}
%\phantomsection
\pdfbookmark[1]{\contentsname}{tableofcontents}
\setcounter{tocdepth}{2} % <-- 2 includes up to subsections in the ToC
\setcounter{secnumdepth}{3} % <-- 3 numbers up to subsubsections
\manualmark
\markboth{\spacedlowsmallcaps{\contentsname}}{\spacedlowsmallcaps{\contentsname}}
\tableofcontents
\automark[section]{chapter}
\renewcommand{\chaptermark}[1]{\markboth{\spacedlowsmallcaps{#1}}{\spacedlowsmallcaps{#1}}}
\renewcommand{\sectionmark}[1]{\markright{\textsc{\thesection}\enspace\spacedlowsmallcaps{#1}}}
%*******************************************************
% List of Figures and of the Tables
%*******************************************************
\clearpage
% \pagestyle{empty} % Uncomment this line if your lists should not have any headlines with section name and page number
\begingroup
    \let\clearpage\relax
    \let\cleardoublepage\relax
    %*******************************************************
    % List of Figures
    %*******************************************************
    %\phantomsection
    %\addcontentsline{toc}{chapter}{\listfigurename}
    %\pdfbookmark[1]{\listfigurename}{lof}
    %\listoffigures

    %\vspace{8ex}

    %*******************************************************
    % List of Tables
    %*******************************************************
    %\phantomsection
    %\addcontentsline{toc}{chapter}{\listtablename}
    %\pdfbookmark[1]{\listtablename}{lot}
    %\listoftables

    %\vspace{8ex}
    % \newpage

    %*******************************************************
    % List of Listings
    %*******************************************************
    %\phantomsection
    %\addcontentsline{toc}{chapter}{\lstlistlistingname}
    %\pdfbookmark[1]{\lstlistlistingname}{lol}
    %\lstlistoflistings

    %\vspace{8ex}

    %*******************************************************
    % Acronyms
    %*******************************************************
    %\phantomsection
    \pdfbookmark[1]{Acronyms}{acronyms}
    \markboth{\spacedlowsmallcaps{Acronyms}}{\spacedlowsmallcaps{Acronyms}}
    \chapter*{Acronyms}
    \begin{acronym}[NANOGrav]
    	\acro{BBN}{Big Bang nucleosynthesis}
    	\acro{BBO}{Big Bang Observatory}
    	\acro{B-DECIGO}{basic DECIGO}
    	\acro{BSM}{beyond the Standard Model}
    	\acro{CMB}{cosmic microwave background}
    	\acro{CP}{charge conjugation parity}
    	\acro{CPTA}{Chinese Pulsar Timing Array}
    	\acro{CURN}{common uncorrelated red noise}
    	\acro{DE}{dark energy}
    	\acro{DECIGO}{DECi-hertz Interferometer Gravitational wave Observatory}
    	\acro{DM}{dark matter}
    	\acro{DMGP}{dispersion measure as a Gaussian process}
    	\acro{DMX}{dispersion measure extension of the timing ephemeris}
    	\acro{dof}{degree of freedom}
    	\acroplural{dof}[dofs]{degrees of freedom}
    	\acro{DS}{dark sector}
    	\acro{DSPT}{dark sector phase transition}
    	\acro{ET}{Einstein Telescope}
    	\acro{EPTA}{European Pulsar Timing Array}
    	\acro{EWPT}{electroweak phase transition}
    	\acro{FLRW}{Friedmann–Lemaître–Robertson–Walker}
        \acro{FOPT}{first-order phase transition}
        \acro{GR}{General Relativity}
        \acro{GUT}{grand unified theory}
        \acro{GW}{gravitational wave}
        \acro{GWB}{gravitational wave background}
        \acro{HD}{Hellings-Downs}
        \acro{InPTA}{Indian Pulsar Timing Array}
        \acro{IPTA}{International Pulsar Timing Array}
        \acro{IR}{infrared}
        \acro{JWST}{James Webb Space Telescope}
        \acro{KMS}{Kubo-Martin-Schwinger}
        \acro{LCDM}[$\Lambda$CDM]{$\Lambda$ cold dark matter}
        \acro{LIGO}{Laser Interferometer Gravitational-Wave Observatory}
        \acro{LISA}{Laser Interferometer Space Antenna}
        \acro{LO}{leading-order}
        \acro{LTE}{local thermal equilibrium}
        \acro{LVK}{LIGO-VIRGO-KAGRA}
        \acro{MCMC}{Markov chain-Monte Carlo}
        \acro{MHD}{ma\-gne\-to\-hy\-dro\-dy\-na\-mic}
        \acro{NANOGrav}{North American Nanohertz Observatory for Gravitational Waves}
        \acro{NCRN}[no-CURN]{no common uncorrelated red noise}
        \acro{NLO}{next-to-leading-order}
        \acro{ODE}{ordinary differential equation}
        \acro{PBH}{primordial black hole}
        \acro{PDE}{partial differential equation}
        \acro{PLI}{power-law integrated}
        \acro{PPTA}{Parkes Pulsar Timing Array}
        \acro{PT}{phase transition}
        \acro{PTA}{pulsar timing array}
        \acro{QCD}{quantum chromodynamics}
        \acro{QFT}{quantum field theory}
        \acro{SMBHB}{supermassive black hole binary}
        \acroplural{SMBHB}[SMBHBs]{supermassive black hole binaries}
        \acro{SGWB}{stochastic gravitational wave background}
        \acro{SM}{Standard Model}
        \acro{SNR}{signal-to-noise ratio}
        \acro{SVT}{scalar, vector and tensor}
        \acro{TOA}{time of arrival}
        \acroplural{TOA}[TOAs]{times of arrival}
        \acro{TT}{transverse-traceless}
        \acro{UV}{ultraviolet}
        \acro{vev}{vacuum expectation value}
        \acro{WIMP}{weakly interacting massive particle}
    \end{acronym}

	\vfill
	%\newpage
	\pdfbookmark[1]{Notation}{notation}
	\chapter*{Notation}
	In this thesis, Einstein's summation convention $\sum_\mu a_\mu b^\mu \equiv a_\mu b^\mu$ and the Dirac slash notation $\slashed{k} \equiv k^\mu \gamma_\mu$ are used. If not stated otherwise, natural units in which $c = \hbar = k_\mathrm{B} = 1$ are employed. Further, space-time indices are written as Greek letters, while spatial vectors are printed in boldface ($\bm{x}$) with components denoted by Latin indices ($x_i$). The flat Minkowski metric is chosen as $\eta_{\mu \nu} \equiv \mathrm{diag} \ba{-1,\, +1,\, +1,\ +1}$, where the $0$-component corresponds to the time variable (\ie $x^\mu = \ba{x^0, \, \bm{x}}$ with $x^0 = t$). Partial derivatives with respect to a generic variable $x$ are abbreviated as $\partial_x$; space-time derivatives read $\partial_\mu = \pd{}{x^\mu} = \ba{\partial_t, \, \partial_i}$, while total derivatives with respect to time are denoted by $\dot{f}(t)=\td{f}{t}$. A particle's four-momentum is defined as $p^\mu = \ba{E, \, \bm{p}}$, such that $p_\mu x^\mu = -Et + \bm{p} \cdot \bm{x}$ and $\Diff{4}p = \diff E \Diff{3}p$.
	
	The Einstein tensor is given by $G_{\mu \nu} \equiv R_{\mu \nu} - \frac{1}{2} R g_{\mu \nu}$ with the Ricci scalar $R \equiv R^{\mu}_{\ ~ \mu}$ and the Ricci tensor $R_{\mu \nu} \equiv R^{\gamma}_{\ ~ \mu \gamma \nu}$, which is, in turn, defined over the Riemann tensor
	\begin{align*}
	R^{\mu}_{~\nu \rho \sigma} \equiv \partial_\rho \Gamma^{\mu}_{\nu \sigma} - \partial_\sigma \Gamma^{\mu}_{\nu \rho} + \Gamma^{\mu}_{\alpha \rho} \Gamma^{\alpha}_{\nu \sigma} - \Gamma^{\mu}_{\alpha \sigma} \Gamma^{\alpha}_{\nu \rho} \, .
	\end{align*}
	The Christoffel symbols
	\begin{align*}
	\Gamma^{\rho}_{\mn} \equiv \frac{1}{2} g^{\rho \sigma} \ba{\partial_\mu g_{\sigma \nu} + \partial_\nu g_{\sigma \mu} - \partial_\sigma g_{\mn}}
	\end{align*}
	can be calculated as space-time derivatives of the metric $g_{\mu \nu}$. To simplify the Einstein equations, the reduced Planck mass $\Mp = \ba{8 \pi G}^{-1/2} = 2.4 \cdot 10^{18} \, \text{GeV}$ is employed, where $G$ %$G = 6.7 \times 10^{-11}\, \text{m}^3 \, \text{kg}^{-1} \, \text{s}^{-2}$
	is Newton's gravitational constant. By default, energies, temperatures, masses, and momenta are all given in units of $\text{eV} = 1.6 \cdot 10^{-19}\, \text{J}$, whereas distances and time intervals are expressed in inverse energy units.
\endgroup
