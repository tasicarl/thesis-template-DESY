%*******************************************************
% Acknowledgments
%*******************************************************
\pdfbookmark[1]{Introduction}{introduction}


\chapter{Introduction}

\begin{flushright}
	\slshape
	Ground Control to Major Tom:\\
	Take your protein pills and put your helmet on\\ \medskip
	--- Space Oddity by \textsc{David Bowie}
\end{flushright}

Gravitational waves were first detected by the \acs{LIGO} collaboration in September 2015~\cite{LIGOScientific:2016aoc}, a century after they were first predicted by Albert Einstein in 1915~\cite{Einstein:1916cc}. This groundbreaking discovery marked a significant breakthrough in physics and astronomy in particular. In 2023, the field saw another leap forward when \acfp{PTA} announced the first detection of a \acf{GWB}~\cite{NANOGrav:2023gor, Reardon:2023gzh, EPTA:2023fyk, Xu:2023wog}. \graffito{The dawn of \acs{GW} cosmology} Until now, \acp{GW} have primarily been used as probes of astrophysical phenomena, such as black hole and neutron star mergers. However, with the \ac{PTA} detection, the field of \ac{GW} cosmology is emerging, opening new avenues for understanding the expansion history and matter content of our universe~\cite{Caprini:2018mtu}.

The impact of the \ac{PTA} discovery on cosmology can be compared to the first observation of the \ac{CMB}. Only the following measurements of the \ac{CMB} anisotropies made precision cosmology possible and for the first time allowed testable statements about the universe up to temperatures of a billion Kelvin (i.e., the MeV scale)~\cite{Planck:2018vyg}. This period corresponds to the epoch of \ac{BBN}, which occurred shortly after the decoupling of neutrinos. The latter happened at a point in time, when the interactions between \graffito{Comparing neutrino decoupling ...} neutrinos and other particles of the \ac{SM} of particle physics could no longer be sustained due to the decreasing temperature of the primordial plasma. More quantitatively, this point in time can be determined by comparing the weak interaction rate $\Gamma_\text{weak}(T)$ with the Hubble expansion rate $H(T)$~\cite{Baumann:2022mni}:
\begin{align}
	\frac{\Gamma_\text{weak}}{H} \sim \frac{G_\text{F}^2 T^5}{T^2 / \Mp} \sim \ba{\frac{T}{\text{MeV}}}^3 \, ,
\end{align}
where $G_\text{F} = 1.17 \cdot 10^{-5} \, \text{GeV}^{-2}$ is the Fermi constant. We find that, as soon as the temperature of the primordial plasma dropped below the MeV scale ($T \lesssim \mathcal{O}(\text{MeV})$), neutrinos could no longer interact with the surrounding \ac{SM} particles and henceforth evolved as a decoupled, free-streaming species.

In fact, the same argument can be repeated for \acp{GW}~\cite{Caprini:2018mtu, Maggiore:2007ulw}: When the gravitational scattering rate 
\begin{align}
	\frac{\Gamma_\text{gw}}{H} \sim \frac{G_\text{N}^2 T^5}{T^2 / \Mp} \sim \ba{\frac{T}{\Mp}}^3
\end{align}
dropped below the Hubble rate, the plasma was sufficiently dilute and cold enough in order \graffito{... to the decoupling of \acp{GW}} to let \acp{GW} propagate freely. From this simple scaling argument we can infer that \acp{GW} decoupled already at extremely high temperatures around the Planck scale, namely around a temperature of $10^{32} \, \text{K}$. Consequently, unlike other messengers from the early cosmos like photons, which only decoupled at a temperature of roughly $3000 \, \text{K}$ during recombination~\cite{Baumann:2022mni}, \acp{GW} could free-stream since the Planck epoch: \acp{GW} only redshifted to lower frequencies and weaker amplitudes, preserving their information about the very early cosmos up to the present day. Now, it is on us to find and interpret these signals.

However, the sheer weakness of the gravitational interaction set by Newton’s constant $G_\mathrm{N} = 6.71 \cdot 10^{-39} \, \text{GeV}^{-2}$ is a mixed blessing for cosmologists: On the one hand, in principle it allows the observation of phenomena up to the Planck epoch as shown above. On the other \graffito{$G_\mathrm{N}$: A mixed blessing} hand, only the most extreme events in which incredible amounts of mass and energy move close to the speed of light, can emit \acp{GW} which are strong enough for us to observe them. Following the recent \ac{PTA} measurements, now we are confronted with the burning question what tremendous events could have sourced the novel \ac{PTA} signal in the nHz frequency band.

The leading hypothesis for the signal's origin is an astrophysical \ac{GWB}: During the hierarchical structure formation, galaxies merged forming larger galaxies and clusters~\cite{NANOGrav:2023pdq}. During that process, supermassive black holes residing at the center of galaxies formed bound systems which, assuming the existence of sufficiently effective mechanisms for dissipating binding energy, later inspiraled and \graffito{Supermassive black hole binaries} merged. Their inspiral emits strong \ac{GW} signals in the nHz frequency band. Still, there are ongoing debates whether such a mechanism for dissipating a sufficient amount of binding energy is realized in nature~\cite{Binney:1987,  Ivanov:1998qk, Haiman:2009te, Amaro-Seoane:2009ucl,  Kocsis:2010xa, Vasiliev:2013az,Vasiliev:2013nha, Vasiliev:2015, Dosopoulou:2016hbg} and whether the amplitude and spectral tilt of the observed \ac{GW} spectrum can be explained by realistic populations of \textit{astrophysical} supermassive black holes~\cite{NANOGrav:2023pdq}. In this thesis, we explore two alternative, cosmological mechanisms in detail: Dark sector phase transitions and the inspiral of \textit{primordial} black holes. 

The study of dark sector phase transitions is not only intriguing as they offer an explanation to the \ac{PTA} signal. Instead, it provides new insights into the yet obscure realms of our universe, whose energy content is dominated by dark energy and cold dark matter. In  the following chapter~\ref{chp:gwcosmo} we will see that the history of our \graffito{Dark sector phase transitions} universe can be understood as a series of consecutive phase transitions. As 95\% of our universe are referred to as \textit{dark} and are hence assumed to interact possibly only gravitationally with other matter, it is reasonable to further assume that also a dark sector (to be defined properly in section~\ref{sec:sectors}) featured a phase transition in the early universe. Such a phase transition could have not only given rise to dark matter but also to a still-observable \ac{GWB}~\cite{Breitbach:2018ddu, Schwaller:2015tja, Caprini:2015zlo}.

Primordial black holes on the other hand, while being an interesting dark matter candidate in their own right~\cite{Carr:2020gox}, could have played a decisive role in the early cosmos: There are observations of quasars at high redshifts ($z > 6$, corresponding to a time less then a billion years after the end of inflation) which challenge conventional models of black hole growth~\cite{Volonteri:2010wz, Volonteri:2021sfo, Shapiro:2004ud, Volonteri:2006ma, Tanaka:2008bv}. A population of \acp{PBH} formed in the early universe, could have provided a seed mechanism for these early supermassive black holes in the \graffito{Primordial black holes} center of the first galaxies. Moreover, these \acp{PBH} could have played an important role in the formation of large-scale structure by acting as starting points of galaxy formation~\cite{Carr:2018rid}. Finally, the formation of \acp{PBH} in the early universe through the collapse of large density fluctuations is directly linked to strong dynamics, e.g.~coming from cosmological phase transitions or domain wall networks~\cite{Gouttenoire:2023naa, Gouttenoire:2023gbn}. The study of \acp{GWB}, phase transitions as well as \ac{PBH} formation together with their eventual inspiral and merger is hence full of intriguing interconnections. This motivates us to also consider inspiraling \acp{PBH} as an origin for the recently observed \ac{PTA} signal in this thesis.

After setting the stage for this thesis in an introduction to \ac{GW} cosmology in chapter~\ref{chp:gwcosmo}, followed by a review of the latest \ac{PTA} results in chapter~\ref{chp:PTAs}, and an in-depth description on how to calculate \ac{GWB} predictions for a given dark sector phase transition in chapter~\ref{chp:pt}, we discuss three key questions of this thesis: First, in chapter~\ref{chp:ptabbn}, we aim \graffito{Outline of this thesis} at answering the question what the odds for a dark sector phase transition explanation of the novel nHz signal are. Second, going beyond the nHz frequency range in chapter~\ref{chp:LISA}, we ask what a similar detection of a \ac{GWB} at the future \ac{GW} observatory \acs{LISA} could teach us about the production of \acs{WIMP} dark matter in the early universe through a freeze-out triggered by a dark sector phase transition. The third question we want to answer in this thesis, in chapter~\ref{chp:pbh}, concerns the conditions under which the \ac{PTA} signal can be interpreted as a \ac{GWB} from inspiraling primordial black holes. We summarize our results and conclude this thesis in chapter~\ref{chp:summary}.