%************************************************
\chapter{Summary} \label{chp:summary}
%************************************************

\begin{flushright}
	\slshape
	The wild ideas of yesterday quickly become today's dogma.\\ \medskip
	--- Sheldon Glashow, Physics Nobel laureate 1979
\end{flushright}

In this thesis, we have explored the dawn of \ac{GW} cosmology, an emerging field that promises to revolutionize our understanding of the early universe. Traditionally, \ac{BBN} has served as the earliest probe of the \ac{LCDM} model, allowing us to make \graffito{The meaning of the \ac{PTA} results for cosmology} precise statements up to temperatures around the MeV scale. However, the recent detection of a \acf{GWB} by \acfp{PTA} has provided us with the first direct probe of much higher temperatures, reaching up to the GeV scale.

Specifically, we have demonstrated in chapter~\ref{chp:ptabbn} that \acfp{DSPT} occurring at around $T_\text{p} \simeq 10 \, \text{MeV}$ could explain the observed \ac{GWB}. In a first step we found that a phase transition within a secluded dark sector is in significant tension with precision observables from \ac{BBN} and the \ac{CMB}: Ignoring constraints \graffito{Cosmological constraints challenge stable \acsp{DSPT}} on the effective number of neutrino species ($N_{\text{eff}}$), one could incorrectly conclude that arbitrarily strong \acp{DSPT} could explain the \ac{PTA} signal (see fig.~\ref{fig:Neff_plot}). However, when including these constraints, we find that only transitions with a strength parameter $\alpha < 0.1$ and extremely small $\beta/H \simeq \mathcal{O}(1-10)$, corresponding to very slow transitions, are favored (see fig.~\ref{fig:tringle-comparison}). Requiring $\beta/H > 3$ in order to have bubbles of sub-Hubble size emitting the \ac{GW} signal yields a reasonably good fit to the data. However, further requiring $\beta/H > 10$ to avoid overestimating the \ac{GWB}, leads to the conclusion that no good match to the \ac{PTA} data can be obtained when cosmological constraints are included within our global fit framework.

Thereafter, we have shown that the fit to the data can be significantly improved, and the \ac{DSPT} interpretation preserved, if the dark sector is not secluded but instead decays into the photon bath before \graffito{\acp{PTA} could have observed a decaying \ac{DSPT}} the onset of \ac{BBN}. Modeling this through the decay of a $5 \, \text{MeV}$ scalar into photons in our global fit, we found that the maximal allowed lifetime of the dark Higgs boson is $\tau_\phi = \mathcal{O}(0.1 \, \text{s})$, which is below the sensitivity of laboratory experiments (see fig.~\ref{fig:dsdecay_logprior}). The central conclusions of this chapter are anticipated to remain robust even when updated with the new 15yr \ac{NANOGrav} data set, replacing the 12.5yr data set that was the most recent available at the time our analysis was conducted.

In the study presented in chapter~\ref{chp:LISA}, we examined the implications of a future detection of a \ac{GWB} with \acs{LISA} when interpreted as a \ac{DSPT}. Specifically, we considered the spontaneous breaking of a $U(1)^\prime$ symmetry, triggering the freeze-out of a fermionic dark matter candidate. We uncovered a previously underappreciated correlation  between the peak frequency of the produced \ac{GW} signal and the relic abundance of dark matter (see fig.~\ref{fig:triangle-unrestr}). \graffito{A \ac{FOPT}-triggered freeze-out emits \acp{GW} in \ac{LISA} band} Fixing the produced \ac{DM} abundance to the observed relic density leads to the prediction of a \ac{GWB} within the \acs{LISA} band, see fig.~\ref{fig:triangle-restr}. We performed extensive scans over the $U(1)^\prime$ model parameter space and showed that this conclusion remains a robust feature which could potentially also be present in other dark sector models.

Further, we investigated in detail how the dark sector transfers energy to the \ac{SM} bath and whether it is justified to assume a common temperature for both sectors. We found that the two sectors quickly thermalize for portal couplings of $\lambda_{h\phi} \gtrsim 10^{-6}$ between \graffito{The correlation is not spoiled for hot \acp{DS}, allowing stronger \acp{GWB}}  the dark Higgs and the visible Higgs, a value small enough to satisfy laboratory constraints (see figs.~\ref{fig:evolution}). Even when treating $\lambda_{h\phi} < 10^{-6}$ and the initial temperature ratio between the two sectors (the latter having a strong impact on the amplitude of the emitted \ac{GWB}) as open parameters, the correlation between the \ac{GWB}'s peak frequency and the relic abundance remains robust (see fig.~\ref{fig:triangle-xi2}).

In our final chapter~\ref{chp:pbh}, we explored the conditions under which the inspiral of \acp{PBH} could account for the \ac{GWB} observed by \acp{PTA}. We found that while the signal can be explained by inspiraling \acp{PBH}, this is only possible if the \acp{PBH} are initially clustered, contrary to previous claims (see fig.~\ref{fig:res_explanation}). \graffito{Inspiraling \acp{PBH} can explain \ac{NANOGrav}} We argue that there exists a part of parameter space in which the signal can be explained while avoiding astrophysical constraints, provided that the \acp{PBH} are formed circumventing $\mu$-distortion limits. This explanation of the \ac{PTA} signal is of particular relevance, as it would not be ruled out in case future \ac{PTA} observations find signs of strong anisotropy in the \ac{GWB} spectrum, which indicate a tension with early universe explanations like phase transitions.

In summary, this thesis has contributed to the burgeoning field of \ac{GW} cosmology. It highlights the crucial role that \ac{BBN} and \ac{CMB} observations will play in interpreting \ac{PTA} results and performing model comparisons in global fit frameworks. \graffito{This thesis was only the start} Looking forward, future observatories like \ac{LISA} will expand our observational capabilities to phenomena occurring as early as picoseconds after the end of inflation, possibly helping us in unraveling the origin of dark matter in our universe.

The interplay between \ac{GW} science and dark matter phenomenology is still in its early stages, with much more to explore. This thesis lays the groundwork for understanding the intimate relationship between violent early universe processes like phase transitions or the inspiral of  \graffito{The road ahead} supermassive primordial black holes and their respective \ac{GW} emission. The key takeaway is clear: gravitational wave backgrounds can provide a unique window to the \textit{dark universe} and offer new insights into the fundamental processes that shaped our cosmos.